\chapter{结论}
\label{chap5}

相对于传统的平滑算法,我们的算法改善了相邻两段路况相差较大的情况下的路况计算。使用仿真数据检验了算法的可行性,在轨迹数据较为稀疏,数据量较大的情况下能有很好的结果。当数据量不足的时候对于右转方向也可以有很好的判断。

算法的不足是直行和左转方向的路况判断依赖轨迹的条数比较多,出现这种状况的原因是交通信号灯的随机性比较大,又占到了轨迹总时间的很大比例,造成了数据波动,需要一定的数据量才能拟合回准确值。

未来改进的方向有以下几个方面:
\begin{itemize}
\item 对算法再做优化,降低路口交通信号灯的影响,使得需求的轨迹数量减少。
\item 对算法做扩展,扩充到多个路口的情况,这时需要考虑几个交通信号灯综合作用的情况。
\item 将道路车辆数量,浮动车占到总数量的比例,道路等级,历史路况等信息综合起来考虑,并加入到算法中。
\end{itemize}
