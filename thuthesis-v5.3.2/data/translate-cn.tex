\chapter{外文资料书面翻译}

\title{应用于实时交通传感应用基于隐马尔科夫模型的在线地图匹配}


\textbf{摘要:} 在许多智能交通系统(ITS)应用程序中,探测车辆的数据来源,一个关键的步骤是准确地将GPS轨迹实时地映射到道路网络。这个过程被称为地图匹配,通常需要考虑数据的噪声和稀疏性,因为(1)高度精确的GPS轨迹信息非常稀有,以及(2)密集轨迹对于实况传输和存储是昂贵的。

我们提出了一种基于对噪声和稀疏性鲁棒的隐马尔可夫模型(HMM)的在线地图匹配算法。我们集中在现有的基于HMM的算法的两个改进:(1)使用最佳局部化策略,可变滑动窗口(VSW)方法,保证在不确定的未来输入下的在线解决方案质量,和(2)使用机器学习的空间,时间和拓扑信息。我们使用在城市和农村地区的公交路线上收集的现场测试数据来评估我们的算法的准确性。此外,我们还研究了在处理实时输入流时精度和输出延迟之间的关系。

在我们对现场测试数据的测试中,VSW在精度和输出延迟方面都优于传统的定位方法。我们的研究结果表明,它可以应用于对延迟要求较为苛刻的应用程序,如交通流量传感。

\section{引言}
由大量车辆收集的实时传感器数据,例如城市地区的出租车和公共汽车,为交通感测,交通事故检测,旅行时间预测[3],车辆管理[4]和路线建议[5],[6]。这些系统的可用性取决于通过可用的地图匹配算法提取的数据的可靠性,地图匹配算法做的就是将GPS轨迹投影到数字地图上的相应路段。

诸如时间戳,位置和速度的信息通常由探测车辆记录。 然而,处理这些大量数据所需的高额到难以负担的存储和带宽导致了现在收集稀疏样本的做法,间隔范围从几十秒到几分钟[7]。此外,已知安装在这些探头上的传感器容易出现各种错误[7],例如不精确的位置和速度测量,重复传输和时间戳不匹配。在诸如交通感测的实时应用中,还要求在未来数据点未给出的时候就执行地上的地图匹配,即算法必须在线。可靠的在线地图匹配算法因此必须考虑这些问题并且保证精度和及时性高的输出。

地图匹配算法被定性为全局或增量/在线。全局算法在生成解之前对整个输入轨迹进行批处理。增量/在线算法采用将输入轨迹划分为更小段并按顺序处理的局部化策略,有时导致次优解。在地图匹配中应用的技术包括几何分析[12],信念函数理论[13],扩展卡尔曼滤波器[14]和隐马尔可夫模型[11] [15] [16]。这些方法的优势和局限性已在[9]中综述。特别地,已经采用基于HMM的算法及其变体[10],[17]用于同时评估实际映射的多个假设的能力,以便找到最终的最大似然解。这些方法已经被证明可以抵抗高度噪声测量,例如来自GSM塔的位置指纹[16],并且它们的精度随着轨迹的时间稀疏度的增加而退化[10],[15],[17]。

我们提出的在线HMM(OHMM)地图匹配算法受到最近基于HMM的方法[10],[15] - [17]的启发。我们解决了在以前的相关工作中没有集中的两个具体问题:(i)需要一个在线算法来管理精度和输出延迟之间的权衡,以及(ii)多个评分函数的融合以估计转换可能性。

大多数现有的增量/在线算法使用简单的定位策略,例如固定滑动窗口和固定深度递归预测。滑动窗口方法简单地将轨迹划分为固定大小的输入序列并且独立地处理它们。较大的窗口尺寸导致更好的精度[10],[17]但更长的输出延迟,反之亦然。递归预测方法将每个点的决定延迟固定数量的步长,以评估未来路径替代方案[18]。这两种方法虽然易于实现,但可能导致次优解决方案和长输出延迟。当地图匹配算法在实时输入流上操作并且需要在短时间窗口内产生输出时,这显然是不可行的。受实时应用需求的驱使,我们提出了一个最优的局部化策略,使用可变滑动窗口(VSW)将输入分成更小的子问题,并证明找到全局最优解。我们的方法在概念上类似于在线维特比算法[19],[20]。此外,我们还开发了VSW的次优变体,其提供对延迟性能的最坏情况保证。在第五节中,我们将使用定长滑动窗口(FSW)方法作为基准来比较这两种方法的性能。

其次,我们推导了一种概率评分机制,其在HMM中的表现和转移概率的建模中结合了各种传感器数据和拓扑信息:(i)GPS坐标(ii)车辆速度(iii)速度限制(iv)推断的车辆前进方向,以及(vi)拓扑约束。我们使用支持向量机(SVM)来学习转移概率函数,而不是选择简单的先验模型[10],[15],[17],然后估计其参数。这种方法的主要优点是它提供了一个数据驱动的框架,用于集成多个过渡评分函数。

我们使用在新加坡包括农村和城市地区的4条公交线路上收集的现场测试数据来评估我们的方法的性能。性能是根据精度和输出延迟标准来测量的。我们的研究结果表明,当操作在实时输入流时,所提出的算法实现了优化或接近最优解,实际上具有低输出延迟。

本文结构如下。第二节根据HMM来制定地图匹配问题。方法将在第三节中描述。我们的实验设置在第四部分解释。结果在第五节中介绍。最后,第六部分总结了我们的贡献,并讨论了未来工作的空间。


\section{地图匹配问题}

\subsection{问题定义}

\textbf{定义1}:轨迹,$T=(t_{n} | n=1,...,N)$,是由车辆收集的N个数据点组成的序列。每个轨迹点由其经度$(t_{n}.lon)$,纬度$(t_{n}.lat)$,速度$(t_{n}.v)$和时间戳$(t_{n}.t)$指定。

\textbf{定义2}:段落,$r=(p_{m} | m=1,..,M)$,是一条M点折线表示的道路段曲线。它由连接顶点的一系列线段$p_{1},…,p_{m}$组成。按顺序,其中每个顶点由其经度和纬度指定。 段落也由其道路宽度(r.w),速度限制(r.v)和双向行驶(r.d)的允许性定义(r.d=\{true,flase\})。

\textbf{定义3}:数字地图,$G=\{r_{k} | k=1,..,K\}$,是一组代表K个段落的集合。

给定轨迹T,地图匹配的目标是找到T每个轨迹点到G中段落的对应关系。

\subsection{HMM方法}
在基于HMM的地图匹配算法中,候选路径被顺序地生成并基于它们的似然性来评估。当遇到新的轨迹点时,解决方案的过去的假设被扩展以考虑新的观测值。在最后阶段的所有候选中,具有最高联合概率的幸存路径然后被选择为最终解。

对于每个轨迹点,我们首先从该数据最有可能被采集的地点识别一组候选道路段落。这些候选中的每一个表示为马尔可夫链中的隐藏状态,并且具有表现概率,其是在候选段是真实匹配的情况下观察GPS点的可能性。直观地,如果点被发现更接近它,​​我们将给段更高的概率。然后,我们计算链中每对相邻隐藏状态的转移概率,使得后者的概率仅取决于前者,因此服从马尔可夫假设。我们的目标是找到具有最高联合表现和传输概率的马尔科夫链上的最大似然路径。该过程如图1所示。

正式地,我们将表现概率表示为p(t | h),其是给定隐藏状态(段)的观察到轨迹点t的概率。 从隐藏状态到隐藏状态的转移概率是f(s,h)。给定一个N个点轨迹,隐藏状态的似然序列,$T=(t_{n} | n=1,…,N)$,隐藏状态的似然序列,$S^{*}=(S_{n}\in A_{n} |n=1,...,N)$满足以下递推关系。

\begin{equation*}
V_{n,h}=p(t_{n}h)\max_{s\in A_{n-1}} \{f(s,h)V_{n-1,s}\}.
\end{equation*}

这里$V_{0,h}=p(t_{0}|h)$和$A_{n}$表示第n阶段的隐藏状态集合 。然后,我们可以从最后一个元素中找到$S^{*}$,$s_{N}=\arg\max_{h\in A_{n}}\{V_{N,h}\}$,向后工作找到最大联合概率序列$S_{N-1}$,...$S_{1}$。 我们在第3-E节将提供一个在线算法找到$S^{*}$。


\section{OHMM地图匹配算法}

\subsection{算法的基本流程}
\begin{itemize}
\item 对于每个轨迹点,找到围绕其半径为50m的所有候选分段。 施加该阈值的原因是双重的:(1)丢弃具有非常低的表现概率(低于$10^-4$的范围)的所有候选,以及(2)避免由于过多候选而导致执行速度的惩罚。

\item 针对每个候选分段(隐藏状态)计算表现概率,而将转移概率分配给在隐藏状态上发生的每个边缘。

\item VSW算法在更新的Markov链上执行回溯,并给出部分求解(如果可用)。 否则,输出将延迟一个阶段。

\item 对下一个轨迹点重复上述过程。 当到达最后一点时,算法终止。
\end{itemize}

\subsection{表现概率}
对于在轨迹点附近找到的每个候选分段,我们用1D高斯函数对其观测概率进行建模,如下所示。

\begin{equation*}
p(observation)=\frac{1}{2w}{\int_{-w}^{w}}\frac{1}{\sqrt{2\pi\sigma_{g}^{2}}}e^{-\frac{(l-d)^{2}}{2\pi\sigma_{g}^{2}}}dl.
\end{equation*}

这里w是指道路r的半宽(w=0.5*r.w),d是指t和r间点到曲线之间的大圆距离,$\sigma_{g}$是GPS误差的估计标准偏差。 虽然GPS误差已知具有非高斯分布,我们采用这种模型是因为它易于实现和在之前的工作被证明有效[10],[15],[17]。 我们的方法是不同的,因为我们也考虑道路宽度。 这将允许我们更好地在路段之间区分,特别是在路口。

此外,基于假设驾驶员不大可能大大超过速度限制,我们引入超速的惩罚机制。 目的是帮助区分从相同交叉点分支出的可能具有不同速度限制的紧密间隔的平行道路。 我们注意到在这些情况下,单独的位置测量不足以区分段落,因为记录的轨迹点可能落在它们之间。 我们定义惩罚函数S如下,

\begin{equation*}
S(v_{t},v_{r})=\frac{v_{r}}{max(0,v_{t}-v_{r})+v_{r}}.
\end{equation*}

这里vt指传感器记录的时间,而vr指该路段的限制速度。如果遵守了速度限制,则$max(0,v_{t}-v_{r})=0$。(即惩罚机制不启动)。结合(2)和(3),我们定义表现概率,p(t | r)如下

\begin{equation*}
p(t|r)=S(v_{t},v_{r})p(oberservation).
\end{equation*}

\subsection{转移概率}
令i,j分别表示归属于两个连续轨迹点$t_{n}$和$t_{n+1}$的一对候选段,其中$i\in A_{n}$、$j\in A_{n-1}$。 我们将内插路径$P_{i\to j}$定义为当从i行进到j时车辆最可能采取的段的序列。 假设驾驶员选择最短路程的路(路径距离短的被选择可能性高),我们可以使用A*路径寻找算法找到插值路径[21]。 对于K个片段的给定序列,$P_{i\to j}$中的$(r_{1},…,r_{k})$,我们设计如下的两个评分函数,

\textbf{方法1}:距离差异函数T测量传感器推测的行进距离和插值路径长度之间的差异,

\begin{equation*}
T(d_{i\to j},D_{i\to j})=\frac{|d_{i\to j}-D_{t\to j}|}{D_{i\to j}},
\end{equation*}

这里$d_{i\to j}=\overline{v}_{i\to j}\Delta t$是车辆在时间间隔,以平均速度$\overline{v}_{i\to j}$行驶的距离,而$D_{i\to j}$是$P_{i\to j}$的路径长度。上面的方法1通过比较其长度与推测(DR)估计评估了假设路径$P_{i\to j}$的可行性。如果$P_{i\to j}$是真实路径,差异会被假定为接近于零。

\textbf{方法2}:选用动量变化函数M测量车辆对于在$P_{i\to j}$中采取的每个路段所引起的平均动量变化,

\begin{equation*}
M(v_{0},v_{1},l_{1},...,v_{k},l_{k})=\frac{\sum_{i}^{K}l_{i}||v_{i}-v_{i-1}||}{\overline{v}_{i\to j}\sum_{i}^{K}l_{i}},
\end{equation*}

其中$(v_{1},..,v_{k})$是$P_{i\to j}$中每个段的车辆的速度矢量,并且$(l_{1},...,l_{k})$是相应的段长度。我们假设矢量幅度随时间从$|v_{1}|$线性变化到$|v_{n}|$ ,而它们的方向与段曲线平行。注意,附加参数v0是从先前转换的终端速度继承的初始速度矢量。通过类似的逻辑,我们将使用$v_{k}$,当前终端速度作为下一转换中的初始速度,等等。图2示出了该概念。上述措施2可以被描述为“平滑因子”,其对由许多突然转弯组成的不可行转移做惩罚。

上面介绍的评分函数T和M为转变$i\to j$提供两种不同的“适合度度量”。这表明,可以通过将这两个度量融合在一起来导出转移概率。 我们使用标记为正确或不正确转换的实例训练支持向量机(SVM)分类器,其中特征向量由测量1和测量2给出的分量分数组成。利用该分类方法,$P_{i\to j}$是输入得分组合属于“正确转换”类别的概率。 我们将在第4-B节更详细地描述训练过程。

\subsection{在线维特比算法}
我们的目标是使用增量法找到全局地图匹配解决方案。 这意味着该算法需要在不知道未来输入的情况下沿着马尔可夫链进行不可逆的在线决策,同时确保部分解在组合时产生全局最优解。 为了实现这一点,我们应用在线动态规划来解决(1)中的递推关系。 关键的见解是,当当前幸存路径在马尔科夫链中的某点(收敛点)收敛时,所有未来的幸存路径将包含直到收敛点的相同子路径。 相关证据可在[19]和[20]中找到。

我们制定我们的OHMM算法的伪码如下。算法1(OHMM地图匹配)递增地处理轨迹点,并且在每个阶段,它输出算法2返回的部分解,如果可用的话。否则,它给出一个空输出,并引发一个延迟阶段。算法2(在线维特比算法)检查在解链中是否存在任何收敛点,并返回到该点的最大似然子序列(如果有的话)。

方便地描述算法1和2在滑动窗口方面的工作原理。当在由窗口覆盖的马尔科夫链中的任何地方找到收敛点时,窗口随着新的轨迹点被处理并从后面收缩而向前扩展。注意,滑动窗口的大小可以根据状态空间的结构而变化,因此名称可变滑动窗口。图3示出了VSW的工作原理。

然而,VSW的一个缺点是不能保证最坏情况的窗口大小;因此在极端情况下输出延迟可以是任意大的。我们通过设置窗口大小的上限来修改算法1,使得当达到阈值时,算法将输出直到当前阶段的最大似然解。我们将标记这个修改的方法有界滑动窗(BVSW)方法。但是与VSW不同,这种方法可能导致次优解决方案。

\section{实验装置}

\subsection{现场测试数据}
使用支持GPS的智能手机,我们收集了在新加坡的4条预定的公共汽车路线的地面实况数据,如图4所示。由于我们关心地图匹配结果的路径精度(将在第6-C节中定义),因此对实际测试路径(地面实况路径)的了解足以使我们验证算法。为了比较不同环境条件下的表现,我们选择了涵盖新加坡农村和城市地区的4条路线。农村路线(R1和R2)涉及较少的转弯,并且主要包括通过开放区域的直线路线,例如高速公路。城市路线(U1和U2)除了更加高度分支外,还包括密集地包含高层建筑的城市街区。 R1,R2,U1和U2的长度分别为36.3km,11.3km,27.3km和32.5km。此外,为了模拟变化的采样频率的轨迹,我们以10秒到5分钟的采样间隔对我们的原始数据(每1-3秒记录一次)进行二次采样。

\subsection{训练和参数估计}

参数$\sigma_{g}$需要在(2)中估计体现概率,并且转换概率需要SVM训练。

我们通过分析我们的地面实际数据的扰动来估计$\sigma_{g}$。 对于每个轨迹点,我们计算该点离其最近路段中心的大圆距离。 然后,基于距离的中值绝对偏差(MAD)计算标准偏差,

\begin{equation*}
\sigma_{g}=1.4826 median_{i}(|d_{i}-median_{j}(d_{j})|).
\end{equation*}

这里$d_{i}$表示单个轨迹点与其匹配段之间的垂直距离。 距离的分布允许我们估计地面真实路径周围的轨迹点的一维扰动。 注意在(7)中,MAD由常数因子1.4826缩放,因为我们假设GPS测量误差是正态分布的。 我们采用MAD方法因为它对抗能够更好地对抗数据集中的异常值。 在[7],[15]中采用了相同的估计标准差的方法。 基于整个数据集,我们获得$\sigma_{g}$= 6.86m。

为了推断转移概率,我们使用3,000个标记的实例训练SVM分类器,其中每个实例对应于正确的转换(类别标记为'1')或不正确的转换(类别标记为'2')。每个实例是2D特征向量 (5)和(6)计算得到的得分值,并且两个分量被缩放为[0,1]。 缩放函数是$(1+x)^{-1}$,其中x是特征的任一分量。 使用网格搜索参数空间和5重交叉验证,我们发现参数的最佳组合为C = 0.25和g = 0.5,其中C是软边际参数,g是径向基函数(RBF)内核参数。 训练结果如图5所示。

\subsection{性能评价}
我们将使用两个性能度量来评估我们的算法:精度和输出延迟。

精度被定义为地面实际路径中正确匹配的轨迹点的分数。当轨迹点被映射到包含在地面实况路径中的任何道路段时,记录正确的匹配。这种准确性的测量避免了惩罚“边界类”,其中点位于路口的正中间。我们注意到,精确地确定收集每个轨迹点的路段是不切实际的,原因在于两个原因:(i)“边界类”可以归因于交叉点的任一出口上的路段,以及(ii)可能的误校准的数字地图。因此,路径精度测量是更合适的评估标准。

输出延迟是算法对每个轨迹点引起的平均输出延迟。它通过在获得匹配结果之前花费的时间来量化。

测试如下进行:

\begin{itemize}

\item 我们对农村和城市测试路线的不同采样间隔的测试轨迹执行地图匹配,范围从3秒到5分钟。 对于每个路由类别,我们聚合两个测试路由获得的结果。

\item 我们比较了三种定位策略在精度和输出延迟方面:VSW,BVSW和FSW。 对于BVSW和FWS,测试了不同的窗口尺寸。 对于每个窗口大小,我们聚集整组测试数据的结果(4个测试路由,采样间隔在10秒到5分钟之间)。

\end{itemize}

\section{结果}

图6展示了城乡测试路线的地图匹配精度的比较。 结果表明,除了大于4分钟的抽样间隔外,农村路线的准确性比城市路线的准确度大约为5%。 以小于1分钟的间隔,两条路线的精度都高于0.9。 在这两种情况下,精度随着采样间隔的增加而恶化。

图7和图8中,虚线表示使用VSW定位策略实现的最优结果。 BVSW方法在w = 8及以上时收敛到0.921的最佳精度。 在所有情况下,FSW方法给出一致的较低精度,并且没有收敛到最优解,即使在w = 20。

在图8中,VSW的平均输出延迟为82s。与FSW相比,它实现了显着更低的延迟,而没有折衷解决方案的最优性。 使用BVSW,在4和以上的窗口大小的延迟性能中没有显着的优点。 这表明马尔可夫链中的大多数决策点发生在达到窗口界限之前。 在FSW的情况下,延迟与窗口大小成比例地增加,但是在某个阈值点之后精度增益减小到几乎为零。




\section{结论和未来工作}

在本文中,我们描述了一种用于地图匹配的在线算法,并分析其在地面实况数据上的性能。我们设计了VSW和BVSW方法来寻找在线解决方案。两者在精度和输出延迟方面都优于先前基于HMM的算法中使用的传统FSW定位策略。我们还开发了一种数据驱动的方法,用于推断在地图匹配过程中融合传感器测量和拓扑信息的转换概率。总而言之,这些方法提供了用于设计基于在线HMM的地图匹配算法的一般框架,其适合于使用浮动车辆数据的实时应用。算法的其它变体可以在估计体现和转移概率时结合附加的传感器数据,例如加速度和高度测量。

对于未来的工作,我们可以探索地图匹配算法的设计与动态参数检测和适应不同的环境设置,如在城市或农村地区,其中GPS准确性可能会变化。传感器信息,例如用于GPS测量的精度(DOP)值的稀释可能被证明对于实现这个目标是有用的。此外,我们建议更好的方法[22]内插轨迹点,而不是假设它们之间的最短路径。对实际行进路径的更好近似可以提高地图匹配精度。