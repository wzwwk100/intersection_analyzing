\thusetup{
  %******************************
  % 注意:
  %   1. 配置里面不要出现空行
  %   2. 不需要的配置信息可以删除
  %******************************
  %
  %=====
  % 秘级
  %=====
  secretlevel={秘密},
  secretyear={10},
  %
  %=========
  % 中文信息
  %=========
  ctitle={基于浮动车轨迹数据的路况分析},
  cdegree={工学学士},
  cdepartment={计算机科学与技术系},
  cmajor={计算机科学与技术},
  cauthor={吴铮},
  csupervisor={向勇 副教授},
  %%cassosupervisor={陈文光教授}, % 副指导老师
  %%ccosupervisor={某某某教授}, % 联合指导老师
  % 日期自动使用当前时间,若需指定按如下方式修改:
  % cdate={超新星纪元},
  %
  % 博士后专有部分
  cfirstdiscipline={计算机科学与技术},
  cseconddiscipline={系统结构},
  postdoctordate={2009年7月——2011年7月},
  id={编号}, % 可以留空: id={},
  udc={UDC}, % 可以留空
  catalognumber={分类号}, % 可以留空
  %
  %=========
  % 英文信息
  %=========
  etitle={An Introduction to \LaTeX{} Thesis Template of Tsinghua University v\version},
  % 这块比较复杂,需要分情况讨论:
  % 1. 学术型硕士
  %    edegree:必须为Master of Arts或Master of Science(注意大小写)
  %             “哲学、文学、历史学、法学、教育学、艺术学门类,公共管理学科
  %              填写Master of Arts,其它填写Master of Science”
  %    emajor:“获得一级学科授权的学科填写一级学科名称,其它填写二级学科名称”
  % 2. 专业型硕士
  %    edegree:“填写专业学位英文名称全称”
  %    emajor:“工程硕士填写工程领域,其它专业学位不填写此项”
  % 3. 学术型博士
  %    edegree:Doctor of Philosophy(注意大小写)
  %    emajor:“获得一级学科授权的学科填写一级学科名称,其它填写二级学科名称”
  % 4. 专业型博士
  %    edegree:“填写专业学位英文名称全称”
  %    emajor:不填写此项
  edegree={Doctor of Engineering},
  emajor={Computer Science and Technology},
  eauthor={Xue Ruini},
  esupervisor={Professor Zheng Weimin},
  eassosupervisor={Chen Wenguang},
  % 日期自动生成,若需指定按如下方式修改:
  % edate={December, 2005}
  %
  % 关键词用“英文逗号”分割
  ckeywords={浮动车数据,低采样率数据 , 路口, 路况估计},
  ekeywords={Floating Car Data, Low Sampling Data, Road Junction,Traffic Situation Evaluation}
}

% 定义中英文摘要和关键字
\begin{cabstract}
随着出租车车载全球定位系统(GPS)装置的普及,浮动车(FCD)轨迹信息越来越容易获得。由于浮动车的轨迹信息有着非常好的城市覆盖率,而且数据量庞大,浮动车数据成为了城市交通路况计算中重要的数据来源。而高密度高准确度的GPS信息非常稀有,而且存储和传输成本高,这也使得研究低采样率的浮动车轨迹信息变得很有意义。

在本文中,我们讨论的是低采样率下的浮动车轨迹用于路况计算的情况,主要考虑了相邻两个数据点之间相隔一个路口的情况。本文中的模型采用线性回归作为主要方法,做了一些细节上的优化,在模拟环境下验证了方法的可行性。

  本文的创新点主要有:
  \begin{itemize}
    \item 低采样率下分析经过路口的浮动车轨迹
    \item 没有使用传统的平滑算法
  \end{itemize}

\end{cabstract}

% 如果习惯关键字跟在摘要文字后面,可以用直接命令来设置,如下:
% \ckeywords{浮动车数据,低采样率数据 , 路口, 路况估计}

\begin{eabstract}
With the popularity of taxi car global positioning system (GPS) devices, floating car data (FCD) is much easier to be obtained. As the trajectory of the floating car has a very good urban coverage and a large amount of data, floating car data has become an important source of data for urban traffic estimation. However, highly precise GPS traces are rarely available and brings a great pressure to both storage and transmission system hence it is meaningful to study the trajectory information of the floating car data with low sampling rate.

In this paper, We discuss the use of floating car data at low sampling rates for urban traffic situation evaluation taking the situation that adjacent two data points separated by an intersection into consideration. The model in this paper use linear regression as the main method and pay attention to the optimization of some details. We use simulation data to verify the feasibility of this method.

There are some points making our system different:
   \begin{itemize}
       \item Junction analysis of floating car data at low sampling rate
       \item Abandon the traditional smoothing algorithm
   \end{itemize}
   
\end{eabstract}

% \ekeywords{Floating Car Data, Low Sampling Data, Road Junction,Traffic Situation Evaluation}
