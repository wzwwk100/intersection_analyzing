\chapter{引言}
\label{chp1}

\section{背景}

随着经济发展,汽车的普及率越来越高,已经成为了是人们最常选择的出行手段之一。而汽车持有率的增加导致了很多交通问题,这也就使得实时路况计算变得十分重要。实时路况计算依赖可靠的实时道路轨迹数据,对于高速公路等简单结构的区域,一般采用固定探测装置来进行路况采集~\cite{yuan2014network,wang2005real}。而对于相对复杂的城市区域,装有GPS(Global Position System)系统的浮动车是最近几年兴起的道路路况计算很好的数据来源。与传统的固定探测装置相比,浮动车不仅部署起来更简单,而且在城市区域拥有非常好的覆盖率,部署的成本也更低。但是高采样率的浮动车轨迹数据很难获得,而且储存和传输成本过高,所以我们主要研究低采样率浮动车轨迹数据下的路况计算~\cite{rahmani2010requirements,brockfeld2007benefits}。

对于复杂的城市区域道路网络来说,各个路段的交叉口是对交通状况产生影响最重要的因素之一,判定路口行驶状况的参数有等待队列长度,车辆停止率等等。在本文中,我们忽略了车辆经过路口时的行驶细节,将信号灯或者其他类似于立交桥的道路切换系统中花费的时间统一计作一个虚拟的路口转向延迟。这样做的好处是可以忽略路口的实际结构,使得模型的可扩展性变得更强。

在我们遇到的实际浮动车轨迹数据中,低采样率的数据(两个相邻点之间相隔超过30秒)占到全部数据中的大多数~\cite{yue2009urban},而且95\%以上的数据相邻两个点之间至少间隔了一个路口,这也就使得研究使用跨路口的低采样率浮动车轨迹数据判断两段路分别路况的算法变得很有意义。

\section{研究目标}

由于城市路段路口之间的间隔较短,使得车辆花费在路口的等待时间占到了很大比例,使得使用传统的路况估计方法得到的结果会和实际结果有着比较大的偏差,所以我们需要设计一个模型和算法估计城市区域路口的转向延迟,使得数据的利用率变高,结果也更接近实际值,提高整个城市路况估计系统的准确度。

\section{文章结构}

根据本文的内容,文章一共分成5个章节:
\begin{itemize}

\item 第一章是引言,主要介绍了论文的背景和目标。
\item 第二章是综述,具体介绍了论文想要处理的问题和相关的工作。
\item 第三章是算法设计,介绍了模型的建立和算法的选择。
\item 第四章是实验,介绍了实验的设计和结果分析。
\item 第五章是结论,总结了算法的优势和不足,以及今后可能的优化方向。

\end{itemize}